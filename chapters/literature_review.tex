\chapter{Literature Review}

\begin{justify}
The literature review should provide a story made of the resources you used in your thesis. Research works presented in the literature review should be related to your work and linked together appropriately.  For each paper/research work/resource you used you have to give a short summary of the work then provide your opinion on how it is related to your work, this should include any critic you have or ideas that you adopted in your work.  The Harvard style is used for referencing resources.
\end{justify}

\section{New Heading}
\begin{justify}
Example: \\
Peggy Johnson defines collection development as “the thoughtful process of developing a library collection in response to institutional priorities and community or user needs and interests” (Johnson 2009, p. 1).  According to Johnson (2009, p. 1), collection development forms part of the broader concept of collection management, which involves “an expanded suite of decisions about weeding, cancelling serials, storage, and preservation”.  In an academic library environment, the selection of titles should primarily support the teaching, learning and research needs of the university staff, students and researchers (University of Western Australia Library 2015).  However, the practice of bundling journal titles into one large all-encompassing package has meant that collection development decisions are now often made on a publisher level, rather than on a title-by-title basis (Ball, cited in Carlson \& Pope 2009, p. 385).  In this sense aggregator, packages are similar in nature to monographic blanket orders, where a library agrees to purchase everything that a particular publisher has published (Thompson, Wilder \& Button 2000, p. 214).  The beauty of these large aggregator packages is that they allow library users to access a vast number of online scholarly resources through the click of a mouse button.
\end{justify}

\clearpage