\chapter{Introduction}

\begin{justify}
This section introduces the topic.  It presents the background to the problem area to provide readers a familiarity with the topic so they can understand your proposal.  A strong background and support of the problem is very important as it gives significance to your work.  Describe the problem or questions your work will address and briefly address the overall solution you propose.  You will go into the problem and solution in more detail later, so do not be long and drawn out in the introduction.  This ensures readers are very clear on the importance, background, objective, and scope of your proposed work.
\end{justify}

\section{Problem Statement}
\begin{justify}
Clearly state the problem that your proposed work will answer.  This statement should be focused and concise.  A problem statement should focus on the issues you plan to solve and provide a realistic scope for the project.  This section can provide more details that were not included in the introduction.  If you can clearly define the specific problems you aim to solve and the parameters that govern them, you have succeeded in this section.
\end{justify}

\section{Thesis Organisation}
\begin{justify}
Provide an overview of the coming chapters of the thesis with a short description of each chapter stating the aim and objective of its content.
\end{justify}

\section{Summary (Optional)}
\begin{justify}
Each chapter might end with a summary that provides a brief detail about the content of the chapter.
\end{justify}

\clearpage
